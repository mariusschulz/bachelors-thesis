\subsubsection{Parsing Statements and Expressions}

Every JavaScript program's abstract syntax tree in ESTree format has a root node of type \code{Program} with a \code{body} property that contains an array of all the main program's statements. The simple JavaScript program at the end of the previous section consists of two statements: a variable declaration and an expression statement. \code{x = x + 1} is an assignment expression and therefore not a statement on its own, but since it appears in statement position, the expression is treated as a statement of type \code{ExpressionStatement}.

The process of traversing the abstract syntax tree and deriving a control flow graph from the statements and expressions encountered can be nicely broken up into many functions, each of which knows how to parse a specific node type. One such function can call another to parse the various components of the node it knows how to parse, thereby creating a hierarchy of nested function calls. For example, the aforementioned program could be parsed by a function call hierarchy similar to the following pseudocode:

\begin{Verbatim}[xleftmargin=0.5cm]
- parseProgram()
    - parseStatements()
        - parseStatement()
            - parseVariableDeclaration()
                - parseVariableDeclarators()
                    - parseVariableDeclarator()
                        - parseInit()
                            - parseExpression()
                                - parseLiteral()
                        - parseIdentifier()
        - parseStatement()
            - parseExpressionStatement()
                - parseExpression()
                    - parseAssignmentExpression()
                        - parseRightHandSide()
                            - parseBinaryExpression()
                                - parseLeftHandSide()
                                    - parseIdentifier()
                                - parseRightHandSide()
                                    - parseLiteral()
                                - parseOperator()
                        - parseLeftHandSide()
                            - parseIdentifier()
                        - parseOperator()
\end{Verbatim}

In most cases, every expression statement adds to the control flow graph a new node with an incoming edge that is annotated with the expression. An exception to that rule, however, are expression statements that wrap expressions of type \code{SequenceExpression}.\footnote{A sequence expression as specified by ESTree \cite{estree-spec} consists of two or more expressions, separated by the comma operator, and evaluates to the value of its last expression.} Rather than adding a single node, an individual node is added for each expression in the sequence, which slightly reduces the complexity of those edge annotations in the control flow graph.
