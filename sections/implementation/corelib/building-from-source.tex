\subsubsection{Building from Source}

The source code for Styx is published under the permissive MIT License \cite{mit-license} and is publicly available on GitHub \cite{styx-github}. To build Styx from source, the Git repository needs to be cloned on a local development machine. By running the following commands, the repository is going to be cloned into the \texttt{styx} folder, which is then made the current working directory:

\begin{minted}[xleftmargin=0.75cm]{text}
$ git clone https://github.com/mariusschulz/styx.git
$ cd styx
\end{minted}

Now that the repository has been set up, the npm dependencies declared in the special \texttt{package.json} file need to be restored. They can be installed via npm by running the \texttt{install} command without any arguments:

\begin{minted}[xleftmargin=0.75cm]{text}
$ npm install
\end{minted}

As described in the ``\nameref{sec:build-tools}'' section, the TypeScript code must be transpiled into plain JavaScript. That is most easily done by calling the TypeScript compiler with \texttt{-p} argument pointing to the directory containing the \texttt{tsconfig.json} configuration file (assuming that the current working directory is the root of the Git repository):

\begin{minted}[xleftmargin=0.75cm]{text}
$ tsc -p src
\end{minted}

To simplify the process of bundling modules using Browserify as much as possible, the Gulp \cite{gulp} build system has been set up. It defines a \code{browserify} task in its \texttt{gulpfile.js} configuration file that is found in the repository root. Additionally, the \texttt{package.json} file defines an npm script called \code{browserify} so that the module bundling can be triggered by running the following command:

\begin{minted}[xleftmargin=0.75cm]{text}
$ npm run browserify
\end{minted}

During development, it is handy to have both the TypeScript compiler and Browserify listen for file changes. When a \texttt{.ts} file is saved within the \texttt{src} folder, the TypeScript compiler will transpile it immediately and save the corresponding \texttt{.js} file within the destination directory. That, in turn, will be picked up by the Browserify listener, which will trigger the module bundling again.

To run the TypeScript compiler in watch mode, call it with the \texttt{-w} flag:

\begin{minted}[xleftmargin=0.75cm]{text}
$ tsc -p src -w
\end{minted}

The browserify listener can be started by running another npm script:

\begin{minted}[xleftmargin=0.75cm]{text}
$ npm run browserify-watch
\end{minted}
